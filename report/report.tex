% Created 2014-10-30 Thu 19:01
\documentclass[11pt]{article}
\usepackage[utf8]{inputenc}
\usepackage[T1]{fontenc}
\usepackage{fixltx2e}
\usepackage{graphicx}
\usepackage{longtable}
\usepackage{float}
\usepackage{wrapfig}
\usepackage[normalem]{ulem}
\usepackage{textcomp}
\usepackage{marvosym}
\usepackage{wasysym}
\usepackage{latexsym}
\usepackage{amssymb}
\usepackage{amstext}
\usepackage{hyperref}
\tolerance=1000
\documentclass{scrartcl}
\usepackage[english]{babel}
\usepackage{graphicx}
\author{Pieter Robberechts, Xavier Goás Aguililla}
\date{Friday, December 12, 2014}
\title{Distributed Systems: GAE}
\hypersetup{
  pdfkeywords={},
  pdfsubject={}
}
\begin{document}

\maketitle

\section*{GAE Exercise 3.2}
\label{sec-3.2}
- At which step of the workflow for booking a car reservation (create quote, collect quotes, confirm) would the indirect communication between objects or components kick in?\\
- Which kind of data is passed between both sides? Does it make sense to persist data and only pass references to that data?

max 150 words

-----------

The creation and collection of quotes isn't a lot of work. They are not stored in the database and can be easily put in a collection.
In addition, the result of creating a quote should be immediately visible to the user. The most work-intensive part is checking whether
a list of quotes can be persisted as reservations in the database. This task is performed by a background worker to avoid delay in the frontend. To do so
we serialize the list of quotes and send it to de default push queue. Next, the backend worker will process the tasks in this queue and
notify the user of success or failure. We store this notification in the database. The user can view his notifications at any
time on the \textit{notifications} page.

Alternatively We could have persisted all quotes that need to be confirmed, and then pass references to the back end. However these quotes are
no longer needed after they’re confirmed (or cancelled) and would waste storage space.

\section*{GAE Exercise 3.3}

Assume a scenario in which two different clients try to confirm a couple of tentative reservations, i.e. their quotes are queued to be processed by the back end.
Both include a tentative reservation to the last available car of a certain car type, so that, assuming correct behaviour of the car rental application, it should
fail to confirm the quotes to one of them.
- Is there a scenario in which the code to confirm the quotes is executed multiple times in parallel, resulting in a positive confirmation to both clients’ quotes?\\
- If so, can you name and illustrate one (or more) possibilities to prevent this bogus behaviour?\\
- In case your solution to the previous question limits parallelism, would a different design of the indirect communication channels help to increase parallelism?
For this question, you may assume that a client will have quotes belonging to one car rental company only.\\

max 150 words

-----------

Google App Engine allows to process several tasks simultaneously. So, if a couple of tentative reservations are waiting in the queue to be
processed the code to confirm the quotes could be executed in parallel, resulting in an illegal state. To avoid this scenario we imlemented 
the confirmation process as a transaction. It was a bit tricky to implement this as GAE requires Datastore operations in a transaction to 
operate on entities in the same entity group if the transaction is a single group transaction, or on entities in a maximum of five entity 
groups if the transaction is a cross-group (XG) transaction. This isn't really a problem as we only have two entity groups (Dockx and Hertz), 
but to make our app scalable we use a different transaction for each set of reservations at a distinct company.

\section*{Application URL}
Our application is deployed at \url{https://dockxandhertz.appspot.com}.

\begin{figure}[h]
  \centering
    \includegraphics[width=\textwidth,height=\textheight,keepaspectratio]{Deployment_Diagram}
\end{figure}

\begin{figure}[h]
  \centering
    \includegraphics[width=\textwidth,height=\textheight,keepaspectratio]{Sequence_Diagram}
\end{figure}

% Emacs 24.3.1 (Org mode N/A)
\end{document}
